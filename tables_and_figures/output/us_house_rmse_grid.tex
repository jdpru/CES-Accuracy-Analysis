\begin{table}[!htbp]
\centering
\footnotesize
\caption{U.S. House RMSE by Respondent Threshold and Competitiveness}
\label{tab:us_house_rmse_grid}
\begin{threeparttable}
\begin{tabular}{lcccccc}
\toprule
& \multicolumn{6}{c}{Winner's Vote Share (\%)} \\
\cmidrule(lr){2-7}
Min. Respondents & 20--40 & 40--60 & 40--80 & 60--80 & 80--90 & 90--100 \\
\midrule
All & 10.0 & 8.9 & 9.3 & 9.5 & 11.2 & 10.4 \\
30+ & 10.0 & 8.8 & 9.1 & 9.3 & 11.2 & 9.9 \\
60+ & 4.3 & 8.2 & 8.5 & 8.8 & 10.3 & 10.6 \\
90+ & - & 7.4 & 7.4 & 7.4 & 10.5 & 13.0 \\
120+ & - & 6.8 & 6.7 & 6.7 & 8.9 & 20.3 \\
\midrule
\multicolumn{7}{l}{\textit{Number of District-Years}} \\
\addlinespace[0.2em]
All & 5 & 1076 & 2578 & 1502 & 157 & 149 \\
30+ & 5 & 1068 & 2538 & 1470 & 150 & 142 \\
60+ & 3 & 915 & 2137 & 1222 & 105 & 85 \\
90+ & 0 & 575 & 1183 & 608 & 39 & 25 \\
120+ & 0 & 216 & 400 & 184 & 8 & 2 \\
\bottomrule
\end{tabular}
\begin{tablenotes}
\footnotesize
\item \textit{Note:} RMSE calculated across all years (2006--2022) for congressional districts meeting both criteria. Winner's vote share indicates the benchmark two-party vote share of the winning candidate.
\end{tablenotes}
\end{threeparttable}
\end{table}