\begin{table}[!htbp]
\centering
\footnotesize
\caption{Trend Significance for Always-Secondary Variables}
\label{tab:always_secondary_trends}
\begin{threeparttable}
\begin{tabular}{lccc}
\midrule
\midrule
Variable & N Years & Slope & p-value \\
\midrule
\multicolumn{4}{l}{\textit{Demographic}} \\
\addlinespace[0.2em]
~~Hispanic Origin & 7 & -0.293 & 0.094 \\
~~Employment Status & 9 & 0.172 & 0.173 \\
~~Family Income & 9 & 0.256** & 0.005 \\
~~Union Membership & 9 & 0.065 & 0.212 \\
~~Veteran Status & 9 & -0.137** & 0.001 \\
~~Residence Duration & 6 & -0.237* & 0.050 \\
\addlinespace[0.3em]
\multicolumn{4}{l}{\textit{Voting Administration}} \\
\addlinespace[0.2em]
~~Voting Method & 9 & 0.127 & 0.075 \\
\addlinespace[0.3em]
\multicolumn{4}{l}{\textit{Candidate Choice}} \\
\addlinespace[0.2em]
~~President & 3 & -0.071 & 0.701 \\
~~U.S. House & 8 & 0.194 & 0.264 \\
~~Attorney General & 8 & -0.684* & 0.021 \\
~~Secretary of State & 8 & -0.495* & 0.012 \\
~~State Senator & 9 & -0.094 & 0.617 \\
~~State Representative & 9 & 0.255* & 0.038 \\
\midrule
\end{tabular}
\begin{tablenotes}
\footnotesize
\item * p $<$ 0.05, ** p $<$ 0.01, *** p $<$ 0.001
\item \textit{Note:} Always-secondary variables are classified as Secondary in all years they appear. Slope is the linear trend coefficient (RMSE per year). Negative slopes indicate improving accuracy.
\end{tablenotes}
\end{threeparttable}
\end{table}